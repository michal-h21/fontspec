% fontspec sample for tex4ht
% compile with:
% htxelatex sample "xhtml, charset=utf-8" " -cunihtf -utf8"
% or
% htlualatex sample "xhtml, charset=utf-8" " -cunihtf -utf8"
% you maybe will need to create that script, see 
% http://tex.stackexchange.com/a/27531/2891
% you can also uncomment line \usepackage[...]{tex4ht}
% and run commands: 
% dvilualatex sample
% tex4ht -cunihtf -utf8 sample
% t4ht sample
\documentclass{article}
%\usepackage[xhtml, charset=utf-8,mathml,]{tex4ht}
%\usepackage[utf8]{luainputenc}
\usepackage[]{fontspec}
%\usepackage{url}
\setmainfont{Latin Modern Roman}
\newfontfamily\greekfont[Script=Greek]{Gentium}
% Unicode math still doesn't work
%\usepackage{unicode-math}
%\setmathfont{Latin Modern Math}

\begin{document}
%	\def\verb|#1|{#1}
	\title{TeX4ht and fontspec sample}
	\author{Michal Hoftich}
	\maketitle
	\section{introduction}

	This is sample document which uses \texttt{fontspec}, converted to 
	\verb|html| usint \TeX 4ht. This document can be converted either with 
	\verb|htxelatex| or \verb|htlualatex|. 

	Normally, \TeX 4ht doesn't support opentype fonts used with 
	\verb|fontspec| because it depends on \verb|tfm| files which are 
	used with traditional \TeX fonts.This package tries to patch 
	\verb|fontspec| to not load opentype fonts. 
	Better way would be to add support for opentype fonts to 
	\TeX 4ht binary, but it probably won't happend in near future, 
	so I tried if this way is possible.

	Because we suppress font processing, we must take care of input 
	and output encoding. For utf8 support, we load configuration files 
	depending on script attributes of font declaring commands. 
	Default loaded script is \emph{Latin}, which should support all diacritics 
	used in languages with latin script:
	%\fontspec[Script=Latin,Language=Czech]{TeX Gyre Termes}

	Příliš žluťoučký kůň úpěl ďábelské ódy
	%\fontspec[Script=Greek]{Linux Libertine O}

	Other supported script is \emph{Greek}:

	\greekfont
	ελληνική γλώσσα

	All fonts with different scripts must be defined in preamble

%	\begin{verbatim}
%\newfontfamily\greekfont[Script=Greek]{Gentium}
%\begin{document}
%Some text
%...
%\greekfont
%	ελληνική γλώσσα
%	\end{verbatim} 
	\section{Instalation}

	This is just proof of concept now, so it may not be good idea 
	to install this package somewhere to your local texmf tree. 
	Just run these commands in some directory

	příliš \verb|zluty| sakra.

	\begin{verbatim}
		git clone git@github.com:michal-h21/fontspec.git
		cd fontspec
		pdftex fontspec.dtx
	\end{verbatim} 
	
	\section{Unicode support}

	For new script support, go to page 
	\verb|http://www.utf8-chartable.de/unicode-utf8-table.pl|, 
	select unicode block with the script in interest, 
	select decimal display format for utf8 encoding and copy 
	the resulting table to some file. 
	Then run included script \verb|f4ht-utftable.lua|:

\begin{verbatim}
texlua f4ht-utftable.lua < filewithtable > utf-scriptname-4ht.tex
\end{verbatim} 

See file \verb|utf-latin-4ht.tsv| for example of input file and 
file \verb|utf-latin-4ht.tex| for resulting output file.

\end{document}
